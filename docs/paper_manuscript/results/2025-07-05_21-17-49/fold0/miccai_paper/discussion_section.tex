
\section{Discussion}

\subsection{When IL Excels Over RL}

Our results identify several conditions under which imitation learning outperforms reinforcement learning in surgical contexts:

\textbf{Expert-Optimal Demonstrations}: When training data represents near-optimal behavior for the evaluation criteria, RL exploration may discover valid but suboptimal alternatives. In surgical domains, expert demonstrations often represent refined techniques developed through years of training and experience.

\textbf{Evaluation Metric Alignment}: When test metrics directly reward similarity to training demonstrations, IL has a fundamental advantage. This alignment is common in medical domains where expert behavior defines the gold standard.

\textbf{Limited Exploration Benefits}: Surgical domains have strong constraints on safe and effective actions. While RL exploration can discover novel approaches, these may be valid but suboptimal for standard evaluation metrics.

\textbf{Data Sufficiency}: With sufficient expert demonstrations, IL can capture the full range of appropriate behaviors without requiring the additional complexity of RL.

\subsection{Implications for Surgical AI}

\textbf{Resource Allocation}: Our findings suggest that research resources might be better allocated to optimizing IL approaches rather than developing complex RL systems for surgical planning tasks.

\textbf{Safety Considerations}: IL approaches inherently stay closer to expert behavior, potentially offering safety advantages in clinical deployment. RL exploration, while potentially discovering novel approaches, introduces uncertainty that may be undesirable in safety-critical contexts.

\textbf{Deployment Readiness}: Simpler IL models are easier to validate, interpret, and deploy in clinical settings compared to complex RL systems with learned reward functions.

\textbf{Domain-Specific Design}: Our results suggest that surgical AI may require different methodological approaches than general-purpose AI domains where RL typically excels.

\subsection{Limitations and Future Directions}

Several limitations should be considered when interpreting our results:

\textbf{Single Dataset Evaluation}: Our results are based on the CholecT50 dataset for laparoscopic cholecystectomy. Different surgical procedures or datasets might yield different conclusions.

\textbf{Expert Test Set}: Our evaluation uses expert-level test data similar to the training distribution. Results might differ when evaluating on sub-expert data or out-of-distribution scenarios.

\textbf{Metric Alignment}: Our evaluation metrics directly reward expert-like behavior. Alternative evaluation criteria focusing on patient outcomes or novel surgical approaches might favor RL methods.

\textbf{Exploration Strategies}: More sophisticated exploration strategies or reward design might enable RL approaches to outperform IL. However, this remains an open research question.

Future work should explore these limitations by: (1) evaluating on diverse surgical datasets and procedures, (2) developing evaluation metrics that capture surgical effectiveness beyond expert similarity, and (3) investigating advanced RL techniques specifically designed for expert domains.
